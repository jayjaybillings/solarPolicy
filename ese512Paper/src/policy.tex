It is the recommendation of this project that the U.S. federal government 
create a strong solar energy market by outfitting and refitting federal
buildings with photovoltaics. Such a policy will ultimately save the government
directly in electricity costs and its carbon footprint, in spite of the the
significant technological challenges surrounding photovoltaics, especially
those related to intermittancy. In addition to the long-term savings to the
government, such a large scale roll out of photovoltaics would drive the price
of the technology lower without subsidies, thereby making solar energy
affordable for small businesses and families and further decreasing the
national carbon footprint.

We recommend, specifically, a two-fold policy applied to all states except
Alaska that
\begin{enumerate}
  \item all new government buildings entering the planning phase in
fiscal year 2017 be constructed with enough photovoltaics installed on their
roofs or on an equivalent square footage of adjacent land to off-set 30\% of
their electricity consumption from the power grid on a sunny day during peak
hours.
  \item all existing federal buildings undergo a refit by 2020 to install
photovoltaics on their existing roofs or on an equivalent square footage of
adjacent land to off-set 20\% of their electricity consumption from the
power grid on a sunny day during peak hours.
\end{enumerate}

We further recommend that the government pursue all types of photovoltaic
technology with the preference to type determined on what is best for the
facility in question and that the required efficiency of the chosen type be
determined through a certification program. The certification program would act
as a quality check to insure that all solar cells produced by the federal
government were within 10\% of the industry average efficiency for a given
type of photovoltaic, thus preventing the government from purchasing poor
equipment. After energy efficiency, the choice of photovoltaic cell must be
limited to those cells that are produced with a miminimal carbon intensity such
that the overall carbon footprint of the federal facility decreases.
