\section{Using Solar Energy to Power a Federal\\ Electric Vehicle Fleet}

\emph{Eric Muckley}







\subsection{Introduction}

In order to reduce its dependence on fossil fuels and reduce GHG emissions, the U.S. must abandon a transportation paradigm that relies on the use of internal combustion engines (ICEs). As world petroleum reserves are depleted and the price of gasoline inevitably rises, it is likely that electric vehicles (EVs) and hydrogen fuel cell vehicles (FCVs) will replace ICE vehicles as dominant modes of personal transportation. Although FCVs exhibit some clear performance advantages over EVs, existing national infrastructure strongly favors the adoption of EVs over FCVs~\cite{eric}{20}. For this reason, the discussion presented here will focus on the adoption of EVs and hybrid EVs rather than FCVs.  


It is well known that large-scale adoptions of both solar energy and EVs are currently limited by technological and economic problems. The intermittency of solar power generation and the lack of a suitable storage solution for excess production, as well as the high cost of photovoltaics (PVs), has prevented solar energy from becoming a significant source of electricity generation in the U.S. Similarly, the strain that large EV fleets place on the electric grid, as well as their high carbon footprint if powered by electricity from coal, have presented roadblocks to wide-scale adoption. In this discussion, it is demonstrated that the increase of solar energy generation at federal facilities presents a unique opportunity for an economically viable, large-scale implementation of electric vehicles (EVs) into the federal vehicle fleet.

There are significant advantages to coupling the large-scale adoption of EVs with an increase in solar energy generation. Storing excess solar energy in EV batteries can help mitigate the energy storage problem posed by solar energy production, and recharging EVs using solar energy is significantly less expensive than refueling using gasoline. Charging EVs with solar energy reduces both the nation's dependence on fossil fuels and the amount of GHGs emitted by the transportation sector. Coupled with the large-scale federal adoption of energy generation from PVs, a shift towards widespread federal EV usage can make both EVs and PVs commercially feasible by increasing demand in EV and PV markets without enacting costly government subsidies.





\subsection{EV Deployment into the Federal Transportation Fleet}

A decision by the U.S. government to take a leadership role in EV adoption would result in a reduction in national GHG emissions, improvement in the country's energy independence, creation of high-quality jobs, and foster economic growth~\cite{eric}{million}. To help establish the U.S. as a leader in EV usage, in 2011 President Barack Obama set a goal to have one million EVs on U.S. roads by 2015. His plan was to support the EV industry through new tax credits, increases in R\&D expenditures, and investment in programs which aim to develop the infrastructure required for large-scale EV adoption~\cite{eric}{million}. However, it is well into 2015 and the U.S. has only 300,000 EVs on the road~\cite{eric}{third}. President Obama's plan to accelerate growth in the EV marketplace has failed because despite subsidies for EV purchases, consumer demand remains low~\cite{eric}{million}. Rather than paying consumers large subsidies, federal agencies have increased EV demand by purchasing EVs for use in their own transportation fleets. Federal purchase of EVs helps expand the domestic EV market by increasing the demand for EV technologies and services, which in turn drives down the cost of EVs and makes them more economically appealing to the general public. More importantly, this approach can be accomplished without imposing costly federal EV subsidies. The U.S. government currently owns or leases over 635,000 automotive vehicles for transportation purposes, which consume the equivalent of about 300~million gallons of gasoline each year, costing roughly \$1.3~billion annually. Only about 0.02\% of this cost is spent on electricity for recharging EVs in the federal fleet~\cite{eric}{fleetreport}. 

Between 2009 and 2013, the U.S. government purchased and leased an average of about 50,000 new federal fleet vehicles per year. Of the 45,000 new vehicles purchased in 2013, over 90\% were gasoline, diesel, or flexfuel-powered~\cite{eric}{fleetreport}. By limiting EV purchase to less than 1\% of its total vehicle fleet, the U.S. government has traditionally withheld support for the growth of domestic EV industries and failed to capitalize on opportunities to reduce GHG emissions and national dependence on foreign oil. 


Besides fostering growth of the domestic EV marketplace, implementing EVs as federal fleet vehicles would also help federal agencies satisfy the guidelines set by Executive Order 13514, which was issued by President Obama in 2009. This order requires federal agencies to reduce GHG emissions, reduce fleet petroleum consumption by 30\% before 2020, and leverage federal purchasing power to promote environmentally-responsible products and technologies~\cite{eric}{fleetemission}. The order also stipulates that 95\% of all federal contracts must meet sustainability requirements. Enacting a plan for EV incorporation into the federal transportation fleet enables significant progress toward achievement of the goals presented by Executive Order 13514, especially in the context of emissions abatement, petroleum usage reduction, and investment in sustainable technologies.

Since Obama began his initiatives to decrease U.S. dependence on ICEs for transportation, the number of EVs in the federal fleet has grown from 57 in 2009 to 4,000 in 2013, and the number of fleet hybrids has increased from 1,800 to 16,000 during the same time period. While this significant increase in EV usage represents important progress towards meeting federal mandates, it also causes problems. Increasing EV usage without implementing a sufficient amount of renewable energy generation means that the electricity used to recharge EV batteries must be produced from fossil fuel-burning power plants. In addition to increasing GHG emissions, this also increases the nation's dependence on fossil fuels. 




\subsection{Emissions Abatement by EV Adoption}

One of the primary drivers for large-scale EV adoption is the effort to reduce pollution and GHG emissions from vehicles. However, it is clear that even widespread EV use will not significantly reduce national GHG emissions if the electricity used to recharge the EVs is produced by burning fossil fuels. This means that increasing the generation capacity of renewable energy sources like solar and wind will have a significant effect on the effectiveness of EV implementation to improve air quality. Since renewable energy generation alone cannot help the U.S. reduce its dependence on oil for transportation, there are clear advantages to developing renewable energy infrastructure in tandem with EV adoption. Only by charging EVs using electricity generated from renewable sources can the U.S. reduce both national GHG emissions and its dependence on fossil fuels.

The effect on air quality of shifting to an EV-based transportation paradigm should not be underestimated. ICE vehicles used for personal transportation are responsible for over 40\% of the nonmethane organic gases (NMOGs), 57\% of the nitrogen oxides (NOs), and 82\% of the carbon monoxide (CO) in urban air pollution~\cite{eric}{emission}. Using the 1995 U.S. emissions standards for ICE vehicles, replacing an ICE vehicle by an EV would result in emissions reductions of NMOGs by 98\%, NOs by 92\%, and CO by 99\% per vehicle~\cite{eric}{abatement}. Decoupling the country's transportation requirements from GHG emissions can play a crucial role in mitigating climate change without imposing costly carbon taxes or other federal transportation legislation~\cite{eric}{workingpaper}. Furthermore, using solar energy to recharge EVs ensures that the vehicle emissions are not merely being transferred to the site of a fossil fuel-burning power plant, but are truly being replaced by a clean renewable energy source. This makes it especially important that increases in EV usage are closely tied to renewable energy production.


\subsection{EVs and Solar Energy Generation}


The development of a large federal EV fleet for transportation at and between federal facilities like military bases and national parks would strongly complement the implementation of solar infrastructure at these locations. This is partly due to the fact that solar investment payback times occur much more quickly when the electricity generated is not used solely for insertion into the grid, but for replacing gasoline to power vehicles~\cite{eric}{sierra}. Generating electricity using PVs, storing it, and consuming it on-site instead of immediately transferring it entirely onto the electricity grid has other advantages as well. Utilities commonly charge electricity producers a fee when that producer adds power to the grid. In fact, the Arizona Public Service attempted to impose a \$50 surcharge on electricity customers who produced their own power using PVs~\cite{eric}{fee}. By consuming electricity generated on-site, federal facilities can avoid grid fees collected by utilities, ultimately improving the long-term return on investment in solar energy generation systems. Since electricity generated by PVs is carried by direct current (DC), charging EVs directly from PV systems without first converting to alternating current (AC) also bypasses the need for power inversion, which typically introduces efficiency loses of at least 10\% each time the power is inverted~\cite{eric}{invert}. This makes direct charging of EV batteries by PV systems at government facilities especially attractive.

Insolation levels and real estate prices require that most utility-scale ($>$1~MW) solar plants are located in deserts far from large population centers where energy is consumed. The long-distance transmission and distribution of power from producers to consumers accounts for over 6\% of the energy lost in the United States each year~\cite{eric}{losses}. Using electricity generated on-site to recharge the batteries of electric vehicles minimizes distribution losses and results in higher net output efficiencies for generation systems. Furthermore, federal fleet vehicles which are not used as heavily as typical consumer vehicles are ideal for purely solar charging, as a small 3~kW system can provide 100\% of the power for a typical EV which drives 1,200~miles per year~\cite{eric}{sierra}.

One of the primary factors which currently limits large-scale solar energy production is the lack of a high-capacity robust energy storage platform~\cite{eric}{solarstore}. By charging electric vehicles directly from PV systems, energy storage can be accomplished using existing EV batteries without implementing other expensive storage solutions. Large-scale solar EV charging is more economical than using smaller modular systems, as a single 22~kWh solar energy car charging port with integrated lithium ion batteries can cost more than \$40,000~\cite{eric}{google, carreports}. The PV-EV recharging approach at government facilities advances multiple goals. It increases the market for EV and PV adoption while offering a storage solution for convenient energy retrieval without introducing loses from transmission through complex distribution networks and multiple power inversions. 


One of the reasons that federal facilities present a unique opportunity for PV-EV coupling is that electricity demand and intensity of vehicle usage undergo significant variations throughout the week. While EVs in the federal vehicle fleet may be driven frequently on weekdays, the majority of fleet vehicles are underutilized on weekends, after work hours, and on holidays. This trend is analogous to the electricity consumption patterns of federal facilities: weekdays during working hours electricity consumption is high, but it is low on weekends, holidays, and after working hours. This weekly pattern exhibits a trend which is highly optimized for electric vehicle charging by PV systems at federal facilities. During high electricity demand times, during peak sunlight hours on the weekdays, the solar power generated can be consumed on-site, without the need for inefficient long-distance distribution or storage. On weekends, when electricity consumption at federal facilities is low, the excess power produced can be used to directly charge EVs, which are underutilized during non-working hours, or can be transferred onto the electric grid.



\subsection{Costs and Benefits Associated with EV Adoption}

It is important to estimate the cost of significantly increasing the share of EVs in the federal transportation fleet. The average EV produced by U.S. automakers costs roughly \$35,000, which is around 40\% higher than similarly sized ICE vehicles if purchased new without federal tax subsidies\cite{eric}{40percent, shahan, 27k}. This means that replacing each ICE vehicle in the federal fleet by a similarly-sized EV will cost roughly \$10,000. Although it is clear that EV adoption requires initial financial investment, the cost represents a relatively small amount of the annual federal fleet budget. Coupled with large-scale solar energy generation at federal facilities, increased EV usage will result in significantly lower federal fleet fuel costs, lower GHG emissions from vehicles, and increased national energy independence due to decreased reliance on petroleum imports.


For an analysis of the economics of the large-scale purchase of EVs as federal fleet vehicles, it is assumed that each EV purchased by the federal government is replacing a conventional gasoline-powered vehicle. The roughly 350,000 gasoline-powered vehicles in the federal vehicle fleet currently consume about 300~million gallons of gasoline per year~cite{fleetreport}. This amounts to an average of about 860~gallons of gas per vehicle each year. Driving a typical ICE vehicle produces an estimated $9 \times 10^{-3}$ metric tons of CO$_2$ per gallon of gasoline consumed~\cite{eric}{IPCC}. If one gallon of gasoline is equivalent in energy to about 34~kWh~\cite{eric}{greet}, and traditional sources of electricity produce around $7\times10^{-4}$ metric tons of CO$_2$ per kWh~\cite{eric}{epacalc}, it can be estimated that powering an EV with electricity from fossil fuel-burning power plants produces roughly twice as much CO$_2$ as driving an ICE vehicle. However, recharging EV batteries with power produced from PVs results in a process which is nearly carbon neutral. This further demonstrates that EV adoption in tandem with increased PV generation is essential for decreasing GHG emissions in the transportation sector.

It is also useful to estimate the solar energy generation capacity required for fully charging electric federal fleet vehicles when electricity demand at federal facilities is low. American electric vehicle batteries have typical storage capacities of around 25~kWh~\cite{eric}{edmond}. If a 3~kW solar energy generation system can provide 100\% of the power for an EV which drives 1,200~miles per year~\cite{eric}{sierra}, and the average federal fleet vehicle drives roughly 7,500~miles per year~\cite{eric}{fleetreport}, it can be estimated that about a 20~kW system is required for the full recharging of each federal vehicle. During times at which solar electricity generation is higher than demand, this excess power can be stored in EVs parked at federal facilities. Although this discussion does not include an official policy recommendation for the adoption of EVs by government facilities, it clearly outlines the benefits associated with adoption of EVs into the federal fleet alongside an increase in solar energy generation. 


In 2013, the U.S. federal vehicle fleet consumed over \$1.3~billion worth of fuel. Cost of gasoline made up roughly 88\% of this cost, about \$1~billion worth, while electricity for EVs represented only about 0.02\% of the cost, roughly \$315,000. If the entire federal fleet consumed about 300~million gallons of gasoline, then each vehicle paid an average of roughly \$3 per gallon~\cite{eric}{fleetreport}. By replacing each federal ICE vehicle with an EV and charging that EV using solar power produced on-site at federal facilities, the federal government effectively saves over \$2,500 per year in fuel costs per vehicle. Only after 4~years does this cost make up the extra \$10,000 investment for purchasing EVs over conventional ICE vehicles. An analysis of the cost of replacing ICE vehicles with EVs in the electric fleet and the resulting reduction in carbon emissions and gasoline consumed is summarized in Table~\ref{table:EVplan}.




\begin{table}[]
\caption{The number of EVs which can be supported by installed solar generation capacity, the amount of money saved in gasoline fuel purchases assuming that the EVs are poweres solely by solar energy, and the emissions reduced by the EV fleet each year, assuming that each EV replaces a conventional ICE vehicle. }
\centering 
\begin{tabular}{c c c c} 
\hline 
Installed  &  Gov't EVs & Gas Savings& CO$_2$ equivalent   \\ 
[ .1 ex]
Capacity (MW) &Supported &  (millions of \$/yr) & (metric tons/yr)   \\ 
[ .1 ex]
\hline
10& 500 & 1.3 & 3,900 \\
50& 2,500  & 6.5 & 19,000  \\ 
100& 5,000  & 13 & 38,700  \\ 
250& 12,500 & 32 & 96,750 \\ 
500& 25,000  & 65 & 193,500 \\ 
\hline
\end{tabular}
\label{table:EVplan}
\end{table}





\subsection{EVs and PVs at Los Angeles Air Force Base: A Case Study}

It is instructive to study a concrete example of a facility which has employed PV systems which utilize EVs as an energy storage platform. Over the last decade, the U.S. Air Force has been especially proficient in incorporating EVs into their vehicle fleets. In 2013, the Air Force announced a program to lease 500 EVs and hybrids, costing \$20 million total~\cite{eric}{cleantech}. The program aims to further research on vehicle-to-grid (V2G) technology, which vehicle batteries to serve as mobile energy storage units which can be charged when electricity demand is low and discharged onto the grid when demand is high.

In November of 2014, Los Angeles Air Force Base (LA AFB) revealed that 100\% of their non-tactical vehicle fleet has been replaced with full EVs or hybrids. This fleet consists of 42 charging stations and vehicles, which include the Nissan Leaf, Ford C-Max, Chevy Volt, and VIA hybrid van, each of which is equipped with V2G technology. The reason that LA AFB has become the leader in V2G demonstration is that LA AFB is a leader in solar energy production, which nicely compliments the energy requirements of EV adoption there~\cite{eric}{v2g}. The EVs at LA AFB can be charged directly from solar energy, which helps alleviate the problem of energy storage from intermittent renewable energy generation. It also ensures that the energy required to operate the vehicles is truly clean, and that the vehicle emissions are not merely being transferred to the site of a fossil-fuel burning power plant.


The 42~vehicle fleet at LA AFB is estimated to be able to generate 700~kW when transferring electricity back into the grid, which is enough to power 140 homes on a summer afternoon~\cite{eric}{v2g}. Although the LA AFB V2G demonstration is currently the largest in the world, similar systems are being developed at Joint Base Andrews in Maryland and Joint Base McGuire-Dix-Lakehurst in New Jersey, and while the 42~vehicle fleet at LA AFB represents the largest federal EV fleet in the country, 8~other states have committed to putting 3.3~million EVs on U.S. roads by 2025~\cite{eric}{WH}. As the large-scale adoption of EV fleets with integrated V2G systems allows EVs to become more feasible energy storage platforms, both PVs and EVs become more economically viable solutions for the nation's energy and transportation requirements.


While the V2G system has proven to be a technologically viable option for solar energy storage, current efforts to enact larger-scale V2G schemes have met roadblocks, primarily in the political and regulatory sectors~\cite{eric}{v2g}. This fact further exemplifies the need for a coherent federal policy which can increase demand for solar energy which in turn encourages the purchase of EVs for both transportation and energy storage solutions.




\subsection{Conclusion}

To help establish and maintain itself as a leader in the adoption of clean technologies, the U.S. must embrace the use of EVs. Although subsidies for EV purchase have not been especially successful in driving growth of domestic EV markets, government purchase of EVs for use in the federal transportation fleet can increase the EV demand necessary for American EV industries to grow. However, even widespread EV usage will not result in reductions in GHG emissions or increase America's energy independence if the electricity used to recharge EVs is produced by burning fossil fuels. 

Implementing large-scale solar energy generation provides a number of unique economically and technologically-favorable opportunities for the adoption of EVs into the federal vehicle fleet. EV batteries offer one faucet of a highly desired storage solution to inherently intermittent solar energy. Charging EVs directly from solar energy generation systems while avoiding full integration into the electricity grid avoids efficiency loses stemming from multiple power inversions and long-distance distribution lines. Coupling solar energy to EV recharging also guarantees that GHG emissions from EVs are not being merely transfered to the site of a fossil fuel-burning power plant, but are truly decreasing. Finally, powering EVs with solar energy is much less expensive than purchasing gasoline for refueling ICE federal fleet vehicles. As the production of solar energy grows at federal facilities, the adoption of EVs into the federal vehicle fleet becomes more economical, convenient, and widespread. New demonstrations of V2G technology at federal facilities have shown that the combination of on-site solar power generation in conjunction with the adoption of grid-integrated EVs enables important opportunities to expand markets in both sectors while increasing domestic EV and PV demand, solidifying the U.S. as a leader in sustainable energy production and environmentally-responsible transportation solutions, and ensuring that problems such as solar energy storage can be dealt with in an innovative and economically-feasible way.

\clearpage
\bibliographystyle{eric}{ieeetr}

\bibliography{eric}{solar_policy_refs}{References}
