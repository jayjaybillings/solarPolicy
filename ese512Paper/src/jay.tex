\section{Social Justice and Solar Energy: How Photovoltaics Can Help}

The United States consumed 97.1 quadrillion BTUs of energy in 2013, nearly 
twenty percent of total global consumption \cite{eiaIntStats}. Likewise, the 
population of the United States is projected to hit nearly 417 million people 
by 2016 \cite{usCensus}. Rising energy costs over the next few decades will 
primarily affect the poor and the middle class since even large increases in 
the price of energy are a small percentage of the wages of the wealthy and 
there are so many lower income people in the United States relative to the 
wealthy. Likewise, if we continue to generate our future energy through 
traditional means it will not only mean that we have increased the income gap, 
but also our carbon footprint significantly because it such growth would be 
driven primarily driven by coal and natural gas. Thus, we need an alternative 
form of energy that is readily available, cheap, and overall quite clean. This 
paper will discuss these issues from the perspective of liberal nationalism and 
describe how photovoltaics may be the answer to providing affordable energy to 
those who need it most in such a framework.

\subsection{An Overview of US Energy}

Energy in the United States is primarily generated by three sources: fossil 
fuels, nuclear reactors and renewable energy \cite{eiaAEO2015}. Fossil fuels 
accounted for 81\% of the energy generated in 2013 with 27\%, 18\% and 36\% for 
\% natural gas, coal and petroleum respectively \cite{eiaAEO2015}. Roughly 74\%
of this energy was was for use as electricity with coal holding a slight 
majority over the other sources in 2013. The total cost of energy is expected 
to increase at a rate of 2.4\% per year between the 2013 and 2040 fiscal years 
while the total energy consumption will only grow at a rate of 0.9\% per year 
over the same period.

Natural gas is produced from either from its own wells or as a by-product from 
oil wells. Texas, Pennsylvania and Louisiana were the top three natural gas 
producing states in 2013, \cite{eiaAEO2015}. Coal is mined, either in deep 
underground mines or through strip mining and Wyoming, West Virginia and Kentucky 
were the top three coal producing states in 2013. Texas also consumed the most 
electricity in 2013, followed by California and Florida.

Renewable sources of energy accounted for 18\% of electricity generated in 2013 
and only 10\% of total energy generated. Wind energy was the dominate form of 
renewable energy, followed by solar energy from both photovoltaics and thermal 
sources. Nuclear energy accounted for 16\% of the total electricity generated 
and 8\% of the total energy generated in 2013. Growth in both renewables and 
nuclear energy as a fraction of the total market is expected to remain mostly 
flat for the period from 2013 to 2040.

Energy for transportation is the largest sector of the energy market followed 
closely by industrial energy use, together totalling nearly 70\% of the market 
\cite{eiaAEO2015}. Energy in the transportation sector is primarily used by 
light vehicles for personal transportation. The industrial sector uses most of 
its energy on the production of bulk chemicals, but growth in that form of 
energy is not expected to continue very much past the 2025 fiscal year 
\cite{eiaAEO2015}. Among residential users, heating, cooling and ventilation 
require the most energy per household, followed by heating water. Installations 
of solar energy for residential users are increasing at a rate of 30\% 
presently, but this will decrease to 6\% after a federal investment tax credit 
expires in 2016 \cite{eiaAE2015}.

The energy generated by the United States produced nearly 5.4 billion metric 
tons of Carbon Dioxide (CO2) in 2013 \cite{eiaAEO2015} across all sources of 
fossil fuel, including those used for vehicle transportation. Fuel for 
transportation represents the largest source of carbon emissions, roughly a 
third, followed by the industrial, residential and commercial sectors. CO2 
emissions from the United States are expected to remain mostly flat from a per 
capita perspective over the period from 2013 to 2040 because natural gas is 
expected to continue to displace goal for electricity generation. This accounts 
for roughly 16\% of all global carbon emissions, second only to China’s roughly 
25\% of global carbon emissions.

Consumers in the United States paid an average of \$0.10 per kWh in January 2015 
\cite{eiaEPMTable} and will use on average 13,246 kWh over the course of the year 
\cite{worldbankEPC}, for a total average electricity cost of roughly \$1325 per 
person. The industry average dividend yields for electric utilities in the
United States is 3.72%, \cite{dividendCom}.

\subsection{Problems with the Growth in Prices}

The economics associated with electricity costs in the United States present 
some problems with respect to justice from a liberal nationalism perspective 
primarily because of the distribution of the costs. If a society is truly 
dedicated to providing for the basic needs of its people, as a liberal nation 
must be \cite{tan}, then it must be dedicated to keeping energy costs low.

Consider that the average per capita income in the United States in 2010 was
\$38,337, \cite{censusHHES}. Electricity represents 3.4\% of the total income 
of each person. However, if the cost increases at 2.4\% per year, then the cost 
of electricity will nearly double to 6.6\% of total income per person by 2040.
Growth in wages could make up the difference, but wages have remained mostly 
flat or decreased, per capita, for 90\% of the US population \cite{pew} in the 
past few decades. This is in contrast to the commonly known fact, among investors 
at least, that electric utility companies represent a good investment because 
their dividend is stable over decades. That is: electric utility companies will 
pay investors 3.72\% per year in dividends, while everyone else must pay an 
additional 2.4\% in electricity costs.

The problem then becomes one of making ends meet for lower income families. Money that would have been saved for education, home repairs and medical bills, all arguably basic needs, will instead be redirected to electricity, another basic need in a modern society. To come up short like this represents a failure of the nation to support its people. On the other end of the spectrum, a wealth investor with \$35,618 invested in a utility company could cover their electricity costs on dividends and make an additional 1.32\% dividend. By 2040 the compounded value of the investment would be worth over \$50,000 even after paying for their electricity for 27 years! Thus in addition to failing to meet the basic needs of the majority of citizens, such a scenario where energy costs increase but wages do not also represents an unjust scenario where the poorer citizens are essentially forced to trade on which basic needs they want and simultaneously support the basic needs of the wealthy. 

One might argue that such a scenario does not represent a failure to meet the basic needs of citizens because they may prioritize what needs they choose to pay. That is not a liberal nationalist view; basic needs must be met, not some met in exchange for others.

There are two simple ways to fix this problem and meet the basic needs of individuals: increase the wages of the American worker or arrest the increase in costs of electricity. The discussion below will address the latter.

\subsection{Problems with the Growth in Carbon Footprint}

One appealing aspect about the future of energy in the United States is that the carbon footprint is expected to remain mostly flat on a per capita basis and to increase at a rate of 0.4\% at most over the 2013 to 2040 time period \cite{eiaAEO2015}. While this represents a significant decrease in overall CO2 output compared to the past and is due in part to burning clean natural gas, it still presents problems from a global justice perspective. In fact, the United States must drastically \textit{decrease} its CO2 emissions to meet its international obligations.

Nations subscribing to a philosophy of liberal nationalism have at least some commitment to cosmopolitan justice \cite{tan}, in which case the United States must consider the effects of its contribution to anthropogenic global warming on other nations. There are several methods by which the amount of obligation could be determined \cite{singer}, but what is at least clear is that the United States much reduce its emissions to meet that obligation to other members of the community.

The energy profile and outlook described above shows that if the market continues on its present trajectory it will not meet these obligations. Thus this plan is insufficient and requires some augmentation. It is not necessary that over the 27 year span of the projections that the emissions from the United States goes to zero, but it would at least ideally start to decrease in that time. The amount of decrease required should be sufficient to meet whichever climate change scenarios are deemed acceptable from either a global justice or a human rights perspective \cite{feldt}. It is clear, however, that whatever method is picked to address the carbon footprint concerns must also address the domestic concerns of the US populace to have affordable electricity that does not increase at a rate greater than the rate at which their wages increase. 

\subsection{Solar Energy to the Rescue}
The two concerns above have obvious answers. First, arresting the inflation in electricity prices in a free market requires a change from a centralized and monopolized generation model to one that mixes distributed energy generation at the residential and commercial level with centralized generation for base-line uses. This works by generating energy on-site at homes and businesses and is sufficient to account for most electricity needs for those who live in places with enough room for the equipment. While this may not totally address the problem for those who live in dense inner city apartments, it may still reduce their costs too.

The second problem - CO2 emissions - can be addressed by increasing the amount of clean, renewable sources of electricity instead of allowing the market to grow by burning more natural gas. Natural gas is only clean in the sense that it releases much less CO2 than coal, but it is not clean compared to solar or wind energy. One might argue that solar and wind are ``not clean’’ because through a lifecycle analysis they require significant amounts of CO2 to build, but it is generally true that \textit{anything} created today is going to have a high carbon footprint because the energy used to create it was generated by fossil fuels.

Solar energy in particular has several advantages over other types of renewable energy, particularly wind. It can be placed in areas where wind turbines can not and even on days when the wind is not blowing, the sun is shining. For example, every free standing home has a roof where cells could be mounted, but not necessarily enough room to install a free-standing windmill big enough to power the whole home. At an efficiency of 20\% for commercially available solar cells and an insolation of 4.37 kWh/m$^2$, \cite{wholesaleSolar}, an average home that consumes 30.3 kWh/day of electricity \cite{eiaFAQ} would only need 346 ft$^2$ of photovoltaic solar cells on the roof facing the sun to generate all of their electricity on a sunny day. 

While current silicon based solar cells require significant amounts of electricity to create, which releases a measurable carbon footprint during construction, they do not continue to produce carbon while they generate electricity. Thus, outfitting homes and businesses with photovoltaics would create an overall reduction in the CO2 emissions of the United States.

\subsection{Building a Photovoltaics Market without Subsidies and the Role of the US Government}

Such a scenario is currently impossible in the United States because the photovoltaic solar market is small. However, pursuing such a scenario by creating a strong solar energy market is a win-win for everyone because those on the lower income scale will be able to purchase solar cells and reduce their electricity costs, wealthy investors will be able to support business cropping up in this area (and therefore make a profit) and the overall CO2 emissions of the United States will decrease. 

However, the catch to creating a market is that the price of solar cells must drop. It does not make sense to create a market that only has the appearance of saving money for lower income residents. It is also hard to imagine that government subsidies are a viable option in today’s political climate and, in fact, the SunShot program from the US Department of Energy was created for that reason \cite{sunshot}. The goal of this program is to drive the cost of installed solar electricity to \$0.06 by 2020 from its current cost of \$0.11 installed.

One workable alternative to both of these is to create a market by executing a series of very large purchases by an actor or actors who can benefit from the purchase while simultaneously requiring enough photovoltaics to drive the cost down due to an economy of scale. One such actor could be the United States federal government. If it was economically possible for the government to benefit by outfitting its new buildings with photovoltaics and refitting some or all of its old buildings, then this would be a substantial win for the American people because it could address the problems presented in this paper \textit{and} decrease government spending. 

An obvious concern about such a strategy would be that the market needs to provide the government with a high quality product. It is extremely unfair to create a market with a substandard product that Americans have to pay for twice! Instead, a certification program would be needed to insure that the solar cells procured by the government met certain technical standards as determined by a qualified body of scientists not affiliated with the photovoltaics industry (i.e. - scientists from national labs or universities). 

The remaining sections of this work will examine the economics and certification of photovoltaics deployed for US government facilities. They will also examine three case studies to further examine the costs and benefits including issues surrounding intermittency, the relationship to electric vehicles and what this plan would look like deployed for a typical government building.

http://www.eia.gov/forecasts/aeo/pdf/0383%282015%29.pdf
http://www.eia.gov/cfapps/ipdbproject/IEDIndex3.cfm?tid=44&pid=44&aid=2
http://www.census.gov/content/dam/Census/library/publications/2015/demo/p25-1143.pdf
http://www.eia.gov/electricity/monthly/epm_table_grapher.cfm?t=epmt_5_6_a 
http://data.worldbank.org/indicator/EG.USE.ELEC.KH.PC 
http://www.dividend.com/dividend-stocks/utilities/electric-utilities/#table=stocktable&column=3&sort=1 
http://www.census.gov/hhes/www/cpstables/032011/perinc/new01_001.htm 
http://www.pewresearch.org/fact-tank/2013/12/05/u-s-income-inequality-on-rise-for-decades-is-now-highest-since-1928/ 
http://www.eia.gov/tools/faqs/faq.cfm?id=97&t=3 
http://www.wholesalesolar.com/solar-information/sun-hours-us-map 
